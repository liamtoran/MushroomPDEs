%\documentclass[11pt]{extarticle}
%Some packages I commonly use.

%A bunch of definitions that make my life easier\

%\setlength{\columnseprule}{1 pt}
%\begin{document}


\section{L'équation de Fisher ou KPP}
\subsection{Préliminaire}
Notre point de départ est l'équation de diffusion:\begin{equation}\dt u = \Delta u  \end{equation}
En plus de la diffusion, considérons des modèles où le taux d'accroissement de $u$ dépend aussi de la densité $u$.\\
Ceci donne les équations de reaction-diffusion:
\begin{equation}\dt u = \Delta u + F(u) \label{eq:ReaDi} \end{equation} 
où F est assez lisse.\\
Il est souvent naturel dans les modèles de considérer $F(u)$ proportionnel à  $u$ pour $u$ petit (``croissance"), et quand $u$ devient proche de 1, l'accroissement $F(u)$ s'arrête: $F(1)=0$ (``saturation").\\
Ces types de modèles ont étés introduits et examinés par les travaux de Fisher[1] et Kolmogorov, Petrovsky et Piscounuv[2] (abrégés KPP).\\
Un exemple d'une telle équation est:

\begin{equation}
	\dt u = \Delta u +u(1-u) \label{eq:KPP}
\end{equation}
qui sera dans la suite étudiée dans le cas 1-dimensionnel en $x$ : $u=u(x,t)$.

\subsection{Réaction}
En observant les solutions constantes en $x$: $u(x,t)=v(t)$ dans \eqref{eq:KPP}, l'équation différentielle ordinaire (EDO ou ODE): \begin{equation}
	\dt v = v - v^2 = F(v)
\end{equation}
est obtenue. \\
Il y a deux équilibres ($F(v)=0)$) pour $v=0$ et $v=1$.Par le théorème de stabilité de Lyapunov, $F'(0)>0$ montre que $v=0$ est instable et $F'(1)<0$ montre $v=1$ est asymptotiquement stable.

\subsection{Réaction-Diffusion}
Dans l’espace $X=C^0_{b,unif}(\mathbb{R},\mathbb{R})$ des fonctions bornées et uniformément continues, il y a existence locale et unicité des solutions de l’équation de Fisher-KPP \eqref{eq:ReaDi}. Grâce à un principe du maximum, il y a aussi existence globale et unicité des solutions.
\newtheorem{theorem}{Théorème}
\begin{theorem}:\\ Existence et Unicité de la solution dans $X$: Soit $U_0 \in X$. Il existe une unique solution de l'équation de Fisher-KPP \eqref{eq:ReaDi}  $U \in C([0,\infty[,X)$ avec condition initiale $U_0$. \end{theorem}
\begin{theorem}:\\ Principe du Maximum: Soit $u_1$ et $u_2$ deux solutions de \eqref{eq:ReaDi}. Si il éxiste $t_0$ tel que $u_1(x,t_0)<u_2(x,t_0) $ $\forall x$ alors $u_1(x,t)<u_2(x,t)$  $ \forall x$ et $\forall t>t_0$ \end{theorem}

\subsection{Solutions en onde plane stationnaire / onde progressive}
Rappellons la définition d'une solution en onde plane stationnaire / onde progressive:
\begin{definition}{Solutions en onde plane stationnaire}\\Une solution en onde plane stationnaire est une solution de la forme $u(x,t)=h(x-st)$ où $c \in \mathbb{R} $.\\ On fera parfois l'abus de notation $u(x,t)=u(x-st)$
\end{definition}
Sous des hypothèses ``faibles" sur $F$, l'équation \eqref{eq:ReaDi}: $\dt u = \Delta u + F(u)$ a alors la propriété surprenante et importante de posséder des solutions en ondes planes stationnaires liant les états d'équilibre $u=1$ (à $-\infty$) et $u=0$ (à $+\infty$).\\
Les hypothèses sur $F$ portent en partie sur le fait que \eqref{eq:ReaDi} doit posséder:\\
- Deux états d'équilibre $u=1$ et $u=0$: $F(0)=F(1)=0$:\\
- Un phénomène de ``croissance" : $F'(0)>0$\\
- Un phénomène de ``saturation" : $F'(1)<0$ \\
\paragraph{Étude des solutions en ondes progressive de \eqref{eq:ReaDi}}:\\
En substituant $u(x,t) = h(x-st) = h(y)$ pour $y=x-st$ dans \eqref{eq:ReaDi}, les équations obtenues sur $h$ sont: \begin{equation} \label{eq:FWS} \left\{
                \begin{array}{ll}
                h''(y)+ sh'(y)+F(h(y))=0 \\
                h(-\infty)= 1 \\  h(+\infty) =0 
                   \end{array}
              \right.
\end{equation} 
qui est une équation elliptique non linéaire. Le problème est donc de trouver $s$ et $h \in C^2$ tels que le système \eqref{eq:FWS} soit vérifié. Le théorème obtenu est le suivant:



\begin{theorem}:\\
Soit $F \in C^1([0,1])$ tel $F(0)=F(1)=0$ et $F\geq 0$. \\
Il existe une vitesse critique $s_*$ telle que $s_*^2 \geq 4F'(0)$ et: \\ \\
- i) $\forall s \geq s_*$, l'équation \eqref{eq:FWS} a une solution $h_s:\mathbb{R} \rightarrow ]0,1[$ de classe $C^3$.\\ Cette solution est unique à translation près. \\
- ii)  $\forall c<c_*$ l'équation \eqref{eq:FWS} n'a pas de solution $h:\mathbb{R} \rightarrow [0,1]$
\end{theorem}
Remarque : Dans le second cas il existe des solutions mais elles ne sont pas confinées dans [0,1] ni dans $\mathbb{R}^+$, ce qui ne fait pas de sens dans une étude de densité de population.

\newpage
\section{Dyamique de Réseaux en Croissance}
Dans cette section et par la suite nous étudions le modèle sur la croissance de réseaux dynamiques branchant, par exemple un champignon, proposé par Rémi Catellier, Yves D'Angelo et Cristiano Ricci, avec rescaling adéquat:
\begin{equation}\label{eq:BDNG}  \left\{
                \begin{array}{ll}
                \dt\mu + \nabla(\mu v) = f(C)(\mu + \rho) -\mu\rho \\
                   \dt(\mu v)+\nabla(\mu v\times v) +T\nabla\mu=-\lambda\mu v+\mu\nabla C-\mu v \rho \\
                 \dt\rho=  F(v) \mu \\
                  \dt C = -b\rho C
                \end{array}
              \right.
\end{equation} 
L'inconnue $\mu$ représente la densité des apex du champignon.\\
L'inconnue $\rho$ représente la densité des hyphes/ du réseau.\\
L'inconnue $v$ représente la vitesse des apex.\\
L'inconnue $C$ représente la concentration des nutriments.\\
Les paramètres $T$, $\lambda$ et $b$ sont des scalaires représentants la température, l'amortissement fluide sur la vitesse des apex, et le taux de consommation des nutriments par le réseau.\\
La fonction $f$ indique l'influence de la concentration de nutriments sur la croissance du champignon. Pour avoir un état stationnaire sur la croissance du champignon,$f(0)=0$ et $f(x)/x$ dans $L^1$ proche de 0 sont imposés.\\
La fonction $F$ représente l'inverse du temps moyen passé par les apex dans un point donné, et est donné par l'expression:
\begin{equation}
	F(V)=(\frac{1}{2\pi T})^\frac{d}{2}\int_{\mathbb{R}^d} |v|\exp(-\frac{|v-V|^2}{2T})dv\end{equation}
où d est la dimension du problème. Ceci est souvent simplifié en substituant $F(V)$ par une constante: $F(V)= F_0$ .\\

\subsection{Explication des équations du système \eqref{eq:BDNG}}
Le champignon est un réseau branchant dynamique qui peut être étudié en deux parties: les apex (pointes du réseau) représentés par leur densité $\mu$ et les hyphes (branches du réseau) représentés par leur densité $\rho$\\
Les lignes du système \eqref{eq:BDNG} représentent:\\
i) La première ligne du système est le bilan de masse sur les apex avec le terme gauche classique $ \dt\mu + \nabla(\mu v) $. Le terme de droite est composé de : - $f(C)(\mu + \rho)$ correspondant a une croissance proportionnelle à la concentration de nutriments du réseau et la masse existante d'apex et d'hyphes, - et un terme $-\mu\rho$ qui correspond à l'anastomose: une pointe qui rencontre une branche va fusionner avec elle et être détruite. Il y a un terme de croissance et un terme de saturation comme pour le modèle KPP.
\\ ii)  La deuxième ligne est le bilan de vitesse avec le terme de gauche classique $ \dt(\mu v)+\nabla(\mu v\times v) $. Le terme $T\nabla\mu$ représente le mouvement brownien suivi par les apex. Le terme $-\lambda\mu v$ représente un amortissement fluide dans la physique du problème. Le terme  $+\mu\nabla C$ représente la tendance des apex à aller vers les milieux de forte concentration. Le terme $-\mu v \rho $ représente la perte de vitesse du à l'anastomose.\\
iii) La troisième ligne correspond à la relation entre les branches et les pointes: la trace laissée par les apex sont les branches.\\
iv) La quatrième ligne décrit l'évolution de la concentration de nutriments: ils sont consommés par les hyphe avec un taux $bC$ où b est une constante positive.
\subsection{Dérivation de l'équation "KPP avec mémoire"}
En faisant tendre $T$ et $\lambda$ vers $+\infty$, avec $\frac{T}{\lambda}=K$ constant, la deuxième ligne de \eqref{eq:BDNG} donne: 
\begin{equation}
	+K\nabla\mu=-\mu v
\end{equation}
En injectant ceci dans la ligne 1 du système, on obtient le système de 3 inconnues suivant:
 \begin{equation} \left\{
                \begin{array}{ll}
                   \dt\mu = K\Delta\mu + f(C)(\mu + \rho) -\mu\rho\\
                 \dt\rho=  F_0 \mu \\
                  \dt C = -b\rho C
                \end{array}
              \right.
\end{equation}
dit "KPP avec mémoire".
%\end{document}