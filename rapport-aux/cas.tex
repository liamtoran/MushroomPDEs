\ifdefined\COMPLETE
\else
\documentclass[11pt]{article}
\input{util} %file containing all the used libraries
\usepackage{tikz}
\begin{document}
\fi
Partout, la zone $S<0$ est invalide (rouge) car $S=\frac{s^2}{T}>0$.\\
Le domaine des vitesses possible est alors l'intersection des domaines valides (verts) pour les trois conditions:

i) Deux cas possibles sur la condition $q(S)<0$:\\


\begin{tabular}{cc}
 La racine de $q/S$ est positive & La racine de $q/S$ est négative \\
\includegraphics[width=.48\textwidth]{Images/qcas1.png} & \includegraphics[width=.48\textwidth]{Images/qcas2.png} \\
\end{tabular}
\\
ii) Deux cas possibles sur la condition $q(S)^2-4\sigma(S)>0$:\\

\begin{tabular}{cc}
Une racine positive. & Trois racines positives. \\
\includegraphics[width=.48\textwidth]{Images/q2cas1.png} & \includegraphics[width=.48\textwidth]{Images/q2cas2.png} \\
\end{tabular}
iii) Deux cas possibles sur la condition $\Delta_A(S) >0$:\\

\begin{tabular}{cc}
Trois racines négatives. & Une racine négative et deux positives.\\
\includegraphics[width=.48\textwidth]{Images/deltacas1.png} & \includegraphics[width=.48\textwidth]{Images/deltacas2.png} \\
\end{tabular}
 \ifdefined\COMPLETE
\else
\end{document}
\fi
